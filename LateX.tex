\usepackage{movie15}
\documentclass{beamer}
\usetheme{Madrid}

\title{Réseau Neuronal Convolutif}
\author{by Bilal K., Lucas R., Renaud D., Alex P. }
%Bilal Kadi, Lucas Rouquairol, Renaud DEROUBAIX, Alex Poldrugo
\centering
\date{Avril 2020}
\begin{document}
\maketitle
\begin{frame}{Contenu}
    \begin{itemize}
        \item Explication d'un Réseau neuronal.
        \item La partie convolutive du réseau neuronal.
        \item Explication de l'entraînement du réseau neuronal.
        \item Partie technique
    \end{itemize}
\end{frame}
\begin{frame}{Contexte}
    \begin{itemize}
        \item Reconnaissance de chiffre sur image
        \item Explication grossière
        \item Orienté informaticien 
        \item Python
    \end{itemize}
\end{frame}
\begin{frame}{Réseau neuronal}
    \framesubtitle{Base : le cerveau et ses neurones}
        \includegraphics[height=5.5cm]{images/cerveauneurone.jpg}
        \centering
\end{frame}
\begin{frame}{Réseau neuronal}
    \framesubtitle{Comment ça marche? Schéma:}
        \includegraphics[height=7.5cm]{images/reseauneuronal1.jpg}
        \centering
\end{frame}
\begin{frame}{Réseau neuronal}
    \framesubtitle{Pour notre exemple: reconnaissance d'un chiffre sur une image}
        \includegraphics[height=7.5cm]{images/reseauneuronal2exemple.jpg}
        \centering
\end{frame}




\begin{frame}
\huge{\centerline{CONVOLUTION ET POOLING}}
\end{frame}


\begin{frame}{Convolution}
    \includegraphics[height = 4cm]{images/Feature_Learning.jpeg}
\end{frame}


\begin{frame}{Convolution : Exemple 12 x 12 Filtre 1}
    \includegraphics[height = 6cm]{images/Ex_sur_4/Ex_4_n°1 (2).png}
\end{frame}

\begin{frame}{Convolution : Exemple 12 x 12 Filtre 1}
    \includegraphics[height = 6cm]{images/Ex_sur_4/Ex_4_n°1 (3).png}
\end{frame}

\begin{frame}{Convolution : Exemple 12 x 12 Filtre 1}
    \includegraphics[height = 6cm]{images/Ex_sur_4/Ex_4_n°1 (4).png}
\end{frame}

\begin{frame}{Convolution : Exemple 12 x 12 Filtre 1}
    \includegraphics[height = 6cm]{images/Ex_sur_4/Ex_4_n°1 (5).png}
\end{frame}

\begin{frame}{Convolution : Exemple 12 x 12 Filtre 1}
    \includegraphics[height = 6cm]{images/Ex_sur_4/Ex_4_n°1 (6).png}
\end{frame}

\begin{frame}{Convolution : Exemple 12 x 12 Filtre 1}
    \includegraphics[height = 6cm]{images/Ex_sur_4/Ex_4_n°1 (7).png}
\end{frame}

\begin{frame}{Convolution : Exemple 12 x 12 Filtre 1}
    \includegraphics[height = 6cm]{images/Ex_sur_4/Ex_4_n°1 (8).png}
\end{frame}

\begin{frame}{Convolution : Exemple 12 x 12 Filtre 1}
    \includegraphics[height = 6cm]{images/Ex_sur_4/Ex_4_n°1 (9).png}
\end{frame}

\begin{frame}{Convolution : Exemple 12 x 12 Filtre 1}
    \includegraphics[height = 6cm]{images/Ex_sur_4/Ex_4_n°1 (10).png}
\end{frame}

\begin{frame}{Convolution : Exemple 12 x 12 Filtre 1}
    \includegraphics[height = 6cm]{images/Ex_sur_4/Ex_4_n°1 (11).png}
\end{frame}

\begin{frame}{Convolution : Exemple 12 x 12 Filtre 1}
    \includegraphics[height = 6cm]{images/Ex_sur_4/Ex_4_n°1 (12).png}
\end{frame}

\begin{frame}{Convolution : Exemple 12 x 12 Filtre 1}
    \includegraphics[height = 6cm]{images/Ex_sur_4/Ex_4_n°1 (13).png}
\end{frame}

\begin{frame}{Convolution : Exemple 12 x 12 Filtre 2}
    \includegraphics[height = 6cm]{images/Ex_sur_4/P1.png}
\end{frame}

\begin{frame}{Convolution : Exemple 12 x 12 Filtre 2}
    \includegraphics[height = 6cm]{images/Ex_sur_4/P2.png}
\end{frame}

\begin{frame}{Convolution : Exemple 12 x 12 Filtre 2}
    \includegraphics[height = 6cm]{images/Ex_sur_4/P3.png}
\end{frame}

\begin{frame}{Convolution : Exemple 12 x 12 Filtre 2}
    \includegraphics[height = 6cm]{images/Ex_sur_4/P4.png}
\end{frame}

\begin{frame}{Convolution : Exemple 12 x 12 Filtre 2}
    \includegraphics[height = 6cm]{images/Ex_sur_4/P5.png}
\end{frame}

\begin{frame}{Convolution : Exemple 12 x 12 Filtre 2}
    \includegraphics[height = 6cm]{images/Ex_sur_4/P6.png}
\end{frame}

\begin{frame}{Convolution : Exemple 12 x 12 Filtre 2}
    \includegraphics[height = 6cm]{images/Ex_sur_4/P8.png}
\end{frame}

\begin{frame}{Convolution : Exemple 12 x 12 Filtre 2}
    \includegraphics[height = 6cm]{images/Ex_sur_4/P9.png}
\end{frame}

\begin{frame}{Convolution : Exemple 12 x 12 Filtre 2}
    \includegraphics[height = 6cm]{images/Ex_sur_4/P10.png}
\end{frame}

\begin{frame}{Convolution : Exemple 12 x 12 Filtre 2}
    \includegraphics[height = 6cm]{images/Ex_sur_4/P11.png}
\end{frame}

\begin{frame}{Convolution : Exemple 12 x 12 Filtre 2}
    \includegraphics[height = 6cm]{images/Ex_sur_4/P12.png}
\end{frame}

\begin{frame}{Convolution : Exemple 12 x 12 Filtre 2}
    \includegraphics[height = 6cm]{images/Ex_sur_4/P13.png}
\end{frame}

\begin{frame}{Pooling}
    \includegraphics[height = 4cm]{images/Feature_Learning.jpeg}
\end{frame}

\begin{frame}{Pooling : Exemple}
    \includegraphics[height = 6cm]{images/pooling.png}
\end{frame}

\begin{frame}
    \begin{itemize}
    
        \item Descente de gradient : méthode de représentation d'un neurone
        \item CNN = (Convolutional neural network) (RNC)
        \item Convolution = Ensemble de filtres qui scannent l'images
        \item + de convolutions = + de précision
        
    \end{itemize}
\end{frame}




%Bilal

\begin{frame}{Explication de l'entraînement d'un réseau neuronal}
\framesubtitle{Quelles sont les différentes étapes de l'entraînement ?}
\begin{itemize}
    \includegraphics[height = 6cm] {images/entrainement1.jpeg}
\end{itemize}
\end{frame}

\begin{frame}{Base de données}
    \item Contient toutes les images, divisée en 2 parties :
    \begin{block}{Base de données d'entraînement}
        Cette partie servira à l'entraînement du CNN
    \end{block}
    \begin{block} {Base de données de validation}
        Cette base de données sert à évaluer le modèle
    \end{block}
    Images étiquetées au préalables selon les catégories
    
\end{frame}
\begin{frame}{L'apprentissage}
\framesubtitle{Initialisation}
    \includegraphics[height = 6cm]{images/initialisation.png}
   
\end{frame}

\begin{frame}{Les couches cachées (hidden layer)}
    \includegraphics[height=6cm]{images/secondlayer.png}
\end{frame}

\begin{frame}{Gradient}
    \begin{itemize}
        \item Au début : résultats aléatoires
        \item On évalue donc le coût
        
    \end{itemize}
    \includegraphics[height = 6cm]{images/cout1.png}
\end{frame}

\begin{frame}{Coût}
    \includegraphics[height = 6cm]{images/cout2.png}
\end{frame}

\begin{frame}{Calcul du gradient}
    \includegraphics[height = 6cm]{images/sigmoid.png}
\end{frame}


\begin{frame}{Propagation du gradient}
    \includegraphics[height = 6cm]{images/propagation.png}
\end{frame}



\begin{frame}{Sur-apprentissage et sous-apprentissage}
  
  \begin{itemize}
      \item Modèle aussi performant sur les données d'entraînement que celles de validation
      \item Modèle performant sur les images d'entraînement mais moins sur les données de validation : sur-apprentissage 
      \item Modèle non performant ni sur l'entraînement ni sur celles de validation : sous-entraînement
  \end{itemize}
\end{frame}


\begin{frame}{Partie technique}
    \item Motivation
    \item API de machine learning
    \item Keras
    \item Exemple de code
\end{frame}
\begin{frame}{Motivation}
    \framesubtitle{Utilisation du CNN}
        \begin{itemize}
            \item voiture auto guidée
            \item reconnaissance facial
            \item soin médical, reconnaissance de maladie sur radio
            \item marquage d'image
        \end{itemize}
\end{frame}
\begin{frame}{Motivation}
\framesubtitle{CNN à la portée de tous}
    \begin{itemize}
        \item mode d'utilisation de plus en plus accessible
        \item sujet de plus en plus répandu 
        \item des cours en ligne facile à apprendre
        \item API facile d'utilisation 
    \end{itemize}
\end{frame}
\begin{frame}{API de machine learning}
    \begin{block}{API?}
        API est l'abréviation de "Application Programming Interface", interface de programmation applicative en français. \\
        D'après Wikipédia: " c'est un ensemble normalisé de classes, de méthodes, de fonctions et de constantes qui sert de façade par laquelle un logiciel offre des services à d'autres logiciels." 
    \end{block}
    \begin{itemize}
        \item Tensorflow
        \item Keras
        \item Theano
        \item Microsoft Cognitive Toolkit
        \item Apache MXNet
        \item PyTorch
    \end{itemize}
\end{frame}
\begin{frame}{Keras}
    Keras est une interface de programmation applicative de haut niveau travaillant par dessus les API TensorFlow, Theano ou CNTK. C'est-à-dire qu'il faut avoir installé l'une de ces librairies dans son environnement de programmation pour utiliser Keras.
\end{frame}
\begin{frame}{Exemple de Code}
\framesubtitle{créer les couches convolutif et neuronals}
    \includegraphics [height = 6cm]{images/CNNBuild.png}
\end{frame}
\begin{frame}{Exemple de Code}
\framesubtitle{entraînement du réseau}
    \includegraphics[height = 4cm]{images/CNNTRAIN.png}
\end{frame}

\begin{frame}{Sources}
    \begin{itemize}
        
        \item https://fr.wikipedia.org/wiki/R\%C3\%A9seau\_neuronal\_convolutif
        \item udemy (deep learning A-Z)
        \item keras.io
        \item https://www.youtube.com/watch?v=8qL2lSQd9L8
        \item[height ] https://www.youtube.com/watch?v=aircAruvnKk&list=PLZHQObOWTQDNU6R1\_67000Dx\_ZCJB-3pi
        \url{https://www.natural-solutions.eu/blog/entrainement-dun-rseau-de-neurones-convolutif}
    \end{itemize}
\end{frame}
\end{document}


